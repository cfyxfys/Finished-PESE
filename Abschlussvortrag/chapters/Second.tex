\section{Projektorganisation}
    F\"ur einen reibungslosen Ablauf haben wir zu Beginn geplant, wie wir unsere Kommunikation gestalten wollen und konnten mit Hilfe unten aufgf\u"hrter Tools eine erfolgsversprechende Roadmap erzielen.
    Die Seminarorganisation schrieb ein regelm\"a\ss{}iges Treffen mit dem Seminarleiter vor um Hilfestellungen und Tipps zu erm\"oglichen. Vom Fachgebiet Fahrzeugtechnik wurde alle 2 Wochen eine Fragerunde zu regelungstechnischen Problemen angeboten. Dieses Treffen diente auch als Austausch mit den anderen Gruppen.
    In unsere Kleingruppe entschieden wir uns dazu, einmal w\"ochentlich zusammen am Auto zu arbeiten um gegenseitiges Helfen zu erm\"oglichen und immer wieder den Stand abzukl\"aren sowie den n\"achsten kurzen Abschnitt zu planen.
    Den Rest der Woche wurde selbstst\"andig an den neu zugeteilten Aufgaben weitergearbeitet.
    Bei akuten Fragen und Problemen fand ebenso eine st\"andige Kommunikation über WhatsApp statt.
    
    \subparagraph{Trello}
    Damit wir die Aufgaben pr\"azise und strukturiert festhalten konnten haben wir uns f\"ur das Organisations-Tool Trello entschieden. Hiermit konnten wir durch Anlegen von Listen unsere Punkte in \glqq ToDo\grqq, \glqq Meeting\grqq, \glqq Gemacht\grqq{} und \glqq Aufgaben f\"ur die Zukunft\grqq{} aufteilen.
    
    \subparagraph{Github}
    F\"ur die Synchronisierung des Programmcodes haben wir das Versionsverwaltungsprogramm GitHub genutzt. Dadurch konnten wir komfortabel den Programmiercode des Fahrzeugs austauschen sowie bei Fehlern auf alte Code-Zust\"ande zur\"uckgreifen.
    \\ \\

\section{Testcases}

	\begin{table}[h]
		\renewcommand{\arraystretch}{1.5}
		\begin{tabular}{|p{0.7\textwidth}|p{0.3\textwidth}|}
			\hline
			Testfall& Bemerkung\\
			\hline
			Das Fahrzeug f\"ahrt rechts und links herum autonom ohne die Fahrbahnbegrenzung zu ber\"uhren.& \\
			\hline
			Das Fahrzeug f\"ahrt auf einer Fahrspur autonom ohne die Spur zu verlassen. & \\
			\hline
			Das Fahrzeug f\"ahrt eine Runde unter 30 s.& \\
			\hline
			Das Fahrzeug kommt von der Fahrbahn ab und muss daraufhin anhalten.& \\
			\hline
			Das Fahrzeug startet mittig auf der Fahrbahn im \ang{45} Winkel zur Fahrbahnbegrenzung & \\
			\hline
			Das Fahrzeug startet mittig auf der Fahrbahn im \ang{60} Winkel zur Fahrbahnbegrenzung& \\
			\hline
			Bei einem Stoppschild h\"alt das Fahrzeug auf der H\"ohe das Schildes f\"ur 2s an und f\"ahrt danach weiter.& \\
			\hline
			Bei einem Geschwindigkeitsschild f\"ahrt das Fahrzeug ab dem Schild mit verringerter Geschwindigkeit weiter.& \\
			\hline
			& \\
			\hline
			Schilderkennung und autonomes Fahren muss gleichzeitig ohne wahrnehmbare Beeintr\"achtigung funktionieren.& \\
			\hline
			
		\end{tabular}
	\end{table}
  
\section{Regelans{\"a}tze}
  
\subsection{1.Ansatz}
  
\subsection{2.Ansatz}
  
\section{Schildererkennung}
  
\section{Auswertung}
  
\section{Konklusion}