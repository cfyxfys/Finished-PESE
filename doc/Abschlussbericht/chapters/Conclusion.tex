\section{Konklusion}

Das Projektseminar stellte sich als eine sehr vielf�ltige und facettenreiche Veranstaltung heraus, bei der das eigentliche Ziel, eine Fahrzeugsteuerung zu programmieren, nur einen Teil der Erkenntnisse darstellt.

Durch den l�ngeren Zeitraum von einem Semester konnten wir uns intensiv mit Ans�tzen des Autonomen Fahrens besch�ftigen und konnten so mit Integrationstests und Projektplanung auf eine spannende und praxisnahe Art und Weise vieles lernen. Insbesondere die Einarbeitung und Verwendung verschiedener Tools in der Software-Entwicklungskette war besonders sinnvoll.

Auch war es sehr interessant, den Umgang mit der Roboter-Middleware ROS zu erlernen. Dass dieses System auch in der Industrie eingesetzt wird, zeigt die Aktualit�t und Relevanz dieses Systems.

Wir konnten auf eine sehr umfangreiche und freundliche Projektbetreuung zur�ckgreifen, sodass wir f�r die verschiedenen Probleme immer einen Ansprechpartner hatten. Sehr hervorzuheben ist auch die freie Gestaltung der Aufgaben. Der genaue Zeitplan war uns �berlassen, die Zusatzaufgabe durften wir uns selbst �berlegen.

Dennoch kamen wir auch mit unseren vergleichsweise primitiven Berechnungen an die Leistungsgrenzen der Recheneinheit. Hier w�re es n�tzlich, die Rechenleistung zu erh�hen, damit Bilder besser verarbeitet werden k�nnten. Auch sind bessere Lenk- und Fahrmotoren wichtig, um Performanzsteigerungen zu erreichen.

Insbesondere hinsichtlich des zuk�nftigen Ziels des Carolo-Cups bietet es sich an, die implementierten Module zu verbessern. Die Regelung l�sst noch viele weitere Verbesserungen zu, w�hrend die Schilderkennung weiter optimiert und erweitert werden kann.

