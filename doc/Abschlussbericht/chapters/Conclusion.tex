\section{Konklusion}

Das Projektseminar stellte sich als eine sehr vielf�ltige und facettenreiche Veranstaltung heraus, bei der das eigentliche Ziel, eine Fahrzeugsteuerung zu programmieren, nur einen Teil der Erkenntnisse darstellt.
Durch den l�ngeren Zeitraum von einem Semester konnten wir uns intensiv mit Ans�tzen des Autonomen Fahrens besch�ftigen und konnten so mit direktem Testen des Codes auf eine spannende und praxisnahe Weise Vieles lernen.
Auch war es sehr Wissen erweiternd, den Umgang mit der Roboter-Middlerware ROS zu erlernen. Dass dieses System auch in der Industrie eingesetzt wird, zeigt wie wichtig und aktuell diese Kompetenzen in der Arbeitswelt sind.
Ebenso erm�glichte das Seminar einen guten Einblick in die verschiedenen Herausforderungen der Projektorganisation. Dadurch lernten wir, die einzelnen Schritte m�glichst effizient aufzuteilen und wurden erfahrener in der Verwendung verschiedener Tools, die uns sehr unterst�tzten.

Wir konnten auf eine sehr umfangreiche und freundliche Projektbetreuung zur�ckgreifen, sodass wir f�r die verschiedenen Probleme immer einen Ansprechpartner hatten.
Sehr hervorzuheben ist auch die freie Gestaltung der Aufgaben. Der genaue Zeitplan war uns �berlassen, die Zusatzaufgabe durften wir uns selbst �berlegen.

Dennoch kamen wir auch mit unseren vergleichsweise primitiven Berechnungen an die Leistungsgrenzen der Recheneinheit. Hier w�re es sch�n, wenn eine schnellere Version verbaut w�rde bzw. eine dezidierte Grafikeinheit zum Einsatz kommen w�rde, damit Bilder besser verarbeitet werden k�nnten, da diese Disziplin im Fokus stand.
Auch kam es immer wieder zum Absturz des Betriebssystems des Fahrzeugs, was zu unerw�nschten Unterbrechungen unserer Arbeit f�hrte.

