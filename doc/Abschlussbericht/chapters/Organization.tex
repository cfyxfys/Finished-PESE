\section{Projektorganisation}
F�r einen reibungslosen Ablauf haben wir zu Beginn geplant, wie wir unsere Kommunikation gestalten wollen, und konnten mit Hilfe der unten aufgef�hrten Tools eine erfolgsversprechende Roadmap erzielen.

Die Seminarorganisation beinhaltete ein regelm��iges Treffen mit dem Seminarleiter, um Hilfestellungen und Tipps zu erm�glichen. Vom Fachgebiet Fahrzeugtechnik wurde alle zwei Wochen eine Fragerunde zu regelungstechnischen Problemen angeboten. Dieses Treffen diente auch als Austausch mit den anderen Gruppen.

In unserer Kleingruppe entschieden wir uns dazu, uns einmal w�chentlich zusammen am Auto zu treffen um gegenseitiges Helfen zu erm�glichen und den n�chsten kurzen Abschnitt zu planen. Den Rest der Woche wurde selbstst�ndig an den neu zugeteilten Aufgaben weitergearbeitet. Bei akuten Fragen und Problemen fand ebenso eine st�ndige Kommunikation �ber WhatsApp statt.

\subparagraph{Trello}
Damit wir die Aufgaben pr�zise und strukturiert festhalten konnten haben wir uns f�r das Organisations-Tool Trello \cite{Trello} entschieden. Hiermit konnten wir durch Anlegen von Listen unsere Punkte in \glqq ToDo\grqq, \glqq Meeting\grqq, \glqq Gemacht\grqq{} und \glqq Aufgaben f�r die Zukunft\grqq{} aufteilen.

\subparagraph{Github}
F�r die Synchronisierung des Programmcodes haben wir das Versionsverwaltungsprogramm Github \cite{Github} genutzt. Dadurch konnten wir komfortabel den Programmcode f�r das Fahrzeug austauschen sowie bei Fehlern auf alte Code-Zust�nde zur�ckgreifen.

Ebenso konnten wir beim Programmieren auf eine Reihe vorhandener Bibliotheken und Funktionen zur�ckgreifen. 

\subparagraph{OpenCV}
Mit OpenCV \cite{OpenCV} konnten wir die Bilder verarbeiten und die ben�tigten Informationen herausfiltern. Auch der Algorithmus zur Bilderkennung basiert auf einem Beispiel einer OpenCV-Demo \cite{findObject}

\subparagraph{quadprog++}
Die Bibliothek quadprog++ \cite{quadprog} stellt eine Optimierungsmethode bereit, welche die Verbesserung des Optimierungsproblems der modellpr�diktiven Regelung erleichtert.

Auch haben wir mit dem Programm doxygen \cite{doxygen} und dem darauf basierenden rosdoc\_lite \cite{rosdoc_lite} eine �bersichtliche Dokumentation unseres Codes mit Kommentaren erstellen k�nnen.