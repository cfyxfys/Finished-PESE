\section{Einf�hrung}
\label{sec:intro}

In der Automobilindustrie wird immer mehr auf autonomes Fahren gesetzt. Hier werden in der Entwicklung st�ndig neue Meilensteine erreicht, sodass ein Fahren ohne eine Person am Steuer immer realistischer wird.

Eine kleine Einf�hrung will hier das Projektseminar Echtzeitsysteme der TU Darmstadt bieten. Anhand eines Modellautos werden Ans�tze realit�tsnaher Algorithmen diskutiert und ausprobiert. Die Kommunikation mit dem Auto geschieht mit Hilfe des Programms Robot Operating System (ROS). Mit dieser Methode testet auch der Kooperationspartner \glqq Fachgebiet Fahrzeugtechnik (FZD)\grqq{} L�sungen am echten Auto, was die Relevanz dieses Projektseminars unterstreicht.

Ein weiterer, nicht weniger wichtiger Schwerpunkt des Seminars ist die erfolgreiche Planung und Durchf�hrung der Gruppenarbeit. Die 5 Mitglieder der einzelnen Gruppen werden aus verschiedenen Vertiefungsrichtungen zugeteilt, damit jeder eigenes Knowhow mitbringen kann. Die verschiedenen Rollen und Verantwortlichkeiten werden im Team erprobt.

\paragraph{Fahrzeug}
Dem Fahrzeug stehen verschiedene Sensoren wie der Ultraschall-, Gyro-, und Hallsensor sowie eine Kinect-Kamera zur Verf�gung um das Umfeld m�glichst realit�tsnah wahrzunehmen. Durch die Kamera kann man sowohl auf ein Farbbild als auch auf ein Tiefenbild zugreifen was eine pr�zise Bildverarbeitung erm�glicht.

\paragraph{Aufgaben}
Die verschiedenen Fortschritte wurden nach einzelnen Aufgaben gestaffelt. Das Abfahren eines Rundkurses anhand der Analyse zweier Linien stellte die Basis- und Pflichtaufgabe dar. Weitere Aufgaben durften selbst vorgeschlagen werden, als m�gliche Beispiele wurden Spurwechsel, Hinderniserkennung und Verkehrsschilderkennung vorgeschlagen.
Unsere Gruppe entschied sich f�r Spurwechsel und Erkennung der Schilder.
Zum Projektumfang z�hlten aber genauso auch das Festlegen von und Pr�fen in Testszenarien um die Funktionalit�t und Zuverl�ssigkeit autonomer Ste�rungen sicherzustellen.  
