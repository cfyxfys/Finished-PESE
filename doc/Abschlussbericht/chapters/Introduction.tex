\section{Einf�hrung}
\label{sec:intro}

In der Automobilindustrie wird zunehmend auf autonomes Fahren gesetzt. Hier werden in der Entwicklung st�ndig neue Meilensteine erreicht, um ein Fahren ohne eine Person am Steuer zu erm�glichen.

Eine kleine Einf�hrung will hier das Projektseminar Echtzeitsysteme der TU Darmstadt bieten. Anhand eines Modellautos werden Ans�tze realit�tsnaher Algorithmen diskutiert und auf einem Modellfahrzeug kleiner Gr��e ausprobiert. Die Kommunikation mit dem Auto erfolgt mit Hilfe der Middleware Robot Operating System (ROS). Mit dieser Methode testet auch der Kooperationspartner \glqq Fachgebiet Fahrzeugtechnik (FZD)\grqq{} L�sungen am echten Auto, was die Relevanz dieses Projektseminars unterstreicht.

Ein weiterer, nicht weniger wichtiger Schwerpunkt des Seminars ist die erfolgreiche Planung und Durchf�hrung der Gruppenarbeit. Die 5 Mitglieder der einzelnen Gruppen werden aus verschiedenen Vertiefungsrichtungen zugeteilt, sodass jeder Teilnehmer eigenes Fachwissen mitbringen konnte. Die verschiedenen Rollen und Verantwortlichkeiten wurden im Team sinnvoll aufgeteilt und in w�chentlichen Treffen effizient und geplant ausgef�hrt.

\paragraph{Fahrzeug}
Dem Fahrzeug stehen verschiedene Sensoren wie ein Ultraschall-, Gyro-, und Hallsensor sowie eine Kinect-Kamera zur Verf�gung, um das Umfeld m�glichst realit�tsnah wahrzunehmen. Durch die Kamera kann man sowohl auf ein Farbbild als auch auf ein Tiefenbild zugreifen, sodass eine pr�zise Bildverarbeitung erm�glicht wird.

\paragraph{Aufgaben}
Die verschiedenen Meilensteine wurden nach einzelnen Aufgaben gestaffelt. Das Abfahren eines Rundkurses anhand der Analyse zweier Linien stellte die Basis- und Pflichtaufgabe dar. Weitere Aufgaben durften selbst vorgeschlagen werden, als m�gliche Beispiele wurden Spurwechsel, Hinderniserkennung und Verkehrsschilderkennung vorgeschlagen. Wir entschieden uns f�r Spurwechsel und Erkennung der Schilder. Zum Projektumfang z�hlten auch das Festlegen von Testszenarien um die Funktionalit�t und Zuverl�ssigkeit des autonomen Betriebs sicherzustellen.  
