\section{Testprotokoll}
Im Folgenden ist das Testprotokoll f�r die durchgef�hrten Integrationstests abgebildet. Dieses wurde an den angegebenen Stichtagen durchgef�hrt, um den Fortschritt mit Meilensteinen festzustellen. \\

\begin{table}[h]
	\renewcommand{\arraystretch}{1.5}
	\begin{center}
	\begin{tabular}{|>{\centering\arraybackslash}p{0.7\textwidth}|>{\centering\arraybackslash}p{0.06\textwidth}|>{\centering\arraybackslash}p{0.06\textwidth}|>{\centering\arraybackslash}p{0.06\textwidth}|}
		\hline
		Testfall& 31.1. & 14.2. & 28.2. \tabularnewline
		\hline
		Das Fahrzeug f�hrt rechts und links herum autonom ohne die Fahrbahnbegrenzung zu ber�hren.& \xmark &\checkmark &\checkmark \tabularnewline
		\hline
		Das Fahrzeug f�hrt auf einer Fahrspur autonom ohne die Spur zu verlassen. & \checkmark & \checkmark & \checkmark \tabularnewline
		\hline
		Das Fahrzeug f�hrt eine Runde unter 30s.& \checkmark  & \checkmark & \checkmark   \tabularnewline
		\space &  $25s$ & $19s$ &  $19s$
		\tabularnewline
		\hline
		Das Fahrzeug kommt von der Fahrbahn ab und muss daraufhin anhalten.& \xmark & \xmark & \checkmark \tabularnewline
		\hline
		Das Fahrzeug startet mittig auf der Fahrbahn im {45} Winkel zur Fahrbahnbegrenzung & \checkmark & \checkmark & \checkmark  \tabularnewline
		\hline
		Das Fahrzeug startet mittig auf der Fahrbahn im {60} Winkel zur Fahrbahnbegrenzung& \xmark & \checkmark & \checkmark \tabularnewline
		\hline
		Bei einem Stoppschild h�lt das Fahrzeug f�r 3s an und f�hrt danach weiter.& \xmark & \checkmark & \checkmark \tabularnewline
		\hline
		Bei einem Geschwindigkeitsschild f�hrt das Fahrzeug ab dem Schild mit verringerter Geschwindigkeit weiter.& \xmark & \checkmark & \checkmark \tabularnewline
		\hline
		Bei einem Fahrbahnverengung Schild wechselt das Fahrzeug 
		die Fahrspur.& \xmark & \checkmark & \checkmark \tabularnewline
		\hline
		Schilderkennung und autonomes Fahren muss gleichzeitig ohne wahrnehmbare Beeintr�chtigung funktionieren.& \xmark & \xmark & \checkmark \tabularnewline
		\hline
		
	\end{tabular}
	\end{center}
\end{table}